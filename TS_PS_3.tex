%Preamble

\documentclass[12pt]{article}
\renewcommand{\baselinestretch}{1.5}
\usepackage{amssymb}
\usepackage{amsfonts}
\usepackage{amsmath}
\usepackage[titletoc,title]{appendix}
\usepackage{bm}
\usepackage[nohead]{geometry}
\usepackage{setspace}
\usepackage[bottom, hang, flushmargin]{footmisc}
\usepackage{indentfirst}
\usepackage{endnotes}
\usepackage{graphicx}
\usepackage{rotating}
\usepackage{natbib}
\usepackage[justification=centering, labelfont=bf, textfont=bf, labelsep=newline]{caption}
\usepackage{pbox}
\usepackage{array,rotating,threeparttable,booktabs,dcolumn,multirow}
\usepackage{enumerate}
\usepackage{hyperref}
\setcounter{MaxMatrixCols}{30}
\newtheorem{theorem}{Theorem}
\newtheorem{acknowledgement}{Acknowledgement}
\newtheorem{algorithm}[theorem]{Algorithm}
\newtheorem{axiom}[theorem]{Axiom}
\newtheorem{case}[theorem]{Case}
\newtheorem{claim}[theorem]{Claim}
\newtheorem{conclusion}[theorem]{Conclusion}
\newtheorem{condition}[theorem]{Condition}
\newtheorem{conjecture}[theorem]{Conjecture}
\newtheorem{corollary}[theorem]{Corollary}
\newtheorem{criterion}[theorem]{Criterion}
\newtheorem{definition}[theorem]{Definition}
\newtheorem{example}[theorem]{Example}
\newtheorem{exercise}[theorem]{Exercise}
\newtheorem{lemma}[theorem]{Lemma}
\newtheorem{notation}[theorem]{Notation}
\newtheorem{problem}[theorem]{Problem}
\newtheorem{proposition}{Proposition}
\newtheorem{remark}[theorem]{Remark}
\newtheorem{solution}[theorem]{Solution}
\newtheorem{summary}[theorem]{Summary}
\newenvironment{proof}[1][Proof]{\noindent\textbf{#1.} }{\ \rule{0.5em}{0.5em}}
\geometry{left=1in,right=1in,top=1.00in,bottom=1.0in}
\DeclareMathOperator*{\argmin}{\arg\!\min}

\begin{document}

\singlespacing

\noindent {EC 607: Time-Series Econometrics \\ Problem Set 3 \\ Melissa Wilson}



\bigskip

\doublespacing

\noindent {\bf Task 2:}

When collecting date, I retrieved seasonally adjusted data for all variables, except the federal funds rate, bond yield, and exchange rate, as these were not available. 

\begin{enumerate}[(a)]
	\item %Part A
	After viewing the time series of all variables, I decide to test GDP, the foreign exchange rate, the federal funds rate, CPI, payroll employees, and the industrial production index for trend stationarity. This is because they appeared to follow a general trend, linear or otherwise, over time. I tested the unemployment rate and bond yields for covariance stationarity. 
	
	\item %Part B
	The dicky-fuller tests yielded the following results.
	
	\ctable[caption={Time Trend Results},label=resultsb,pos=hbp!]{lccc}{}{
		\FL
		\cmidrule{2-3}\cmidrule{5-6} 
		Variable & Test Statistic & Significance \ML
		GDP & -1.90 & None \NN
		Employed & -2.15 & None \NN
		CPI & -3.58 & 5\% \NN
		FFR & -2.81 & None \NN
		IPI & -2.37 & None \NN
		Exchange Rate & 0.17 & None \LL
		}
		
	\ctable[caption={Covariance Stationary Results},label=resultsc,pos=hbc!]{lccc}{}{
		\FL
		\cmidrule{2-3}\cmidrule{5-6} 
		Variable & Test Statistic & Significance \ML
		Unemployment Rate & -2.96 & 5\% \NN
		Bond Yield & -3.71 & 5\% \LL
		}
		
	\item %Part C
	After looking at the first difference time trends, I decided to test the first difference of the exchange rate as a time trend and all other variables as covariance stationary trends.
	
	\item %Part D
	The following results are for the first difference series.
	ctable[caption={ Covariance Stationary First Difference Results },label=resultsb,pos=hbp!]{lccc}{}{
		\FL
		\cmidrule{2-3}\cmidrule{5-6} 
		Variable & Test Statistic & Significance \ML
		dGDP & -5.62 & 1\% \NN
		dEmployed & -4.93 & 1\% \NN
		dFFR & -5.29 & 1\% \NN
		dIPI & -6.59 & 1\% \LL
		}
	
	\item %Part E
	
	\item %Part F

\end{enumerate}

\end{document}















